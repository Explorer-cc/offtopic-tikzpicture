\documentclass[tikz,border=5pt]{standalone}
\usepackage[fontset=fandol]{ctex}
\usepackage[tt=false]{libertine}
%%%%%%%%%%%% Part I %%%%%%%%%%%%%%%%%
% https://tex.stackexchange.com/a/25241/322482
\usepackage{contour}
\contournumber{100}  %number of copies
\def\nn{42}
\def\radius{15cm}
\def\deltaangle{\fpeval{360/\nn}}

%%%%%%%%%%%% Part II %%%%%%%%%%%%%%%%%
% https://tex.stackexchange.com/q/755422/322482
\def\totalwidth{9.5}
\def\totalheight{3.5}
\def\mysep{0.4}
\def\splitratio{0.98}

\newcommand\recursivenodes[6]{%
    % #1: Depth
    % #2: Width
    % #3: Height
    % #4: X coordinate
    % #5: Y coordinate
    % #6: Scale
    \begingroup
    \pic[scale=#6,transform shape] at (#4,#5) {offtopic};
    \ifnum#1>1\relax
        \pgfmathtruncatemacro{\nextdepth}{#1-1}
        \pgfmathsetmacro{\xoffset}{(.5-.25*\splitratio)*#2}
        \pgfmathsetmacro{\nextwidth}{.5*\splitratio*#2}
        \edef\tmph{#3}
        \pgfmathsetmacro{\nextheight}{.5*\splitratio*\tmph}

        \pgfmathsetmacro{\nexty}{#5 - .5 * \tmph - .5 * \nextheight - \mysep * \nextdepth * .15 }
        \edef\tmpx{#4}

        \pgfmathsetmacro{\leftx}{\tmpx - \xoffset}
        \pgfmathsetmacro{\rightx}{\tmpx + \xoffset}

        \pgfmathsetmacro{\nextscale}{.5*\splitratio*#6}
        % Left subtree
        \recursivenodes{\nextdepth}{\nextwidth}{\nextheight}{\leftx}{\nexty}{\nextscale}
        % Right subtree
        \recursivenodes{\nextdepth}{\nextwidth}{\nextheight}{\rightx}{\nexty}{\nextscale}
    \fi
    \endgroup
}

% https://tex.stackexchange.com/a/755485/322482
\makeatletter
\newcommand{\calcscaling}{
  \pgfgettransformentries{\tmpscaleA}{\tmpscaleB}{\tmpscaleC}{\tmpscaleD}{\tmp}{\tmp}%
  \pgfmathsetmacro{\scalingfactor}{sqrt(abs(\tmpscaleA*\tmpscaleD-\tmpscaleB*\tmpscaleC))*sqrt(abs((\pgf@xx/1cm)*(\pgf@yy/1cm)-(\pgf@xy/1cm)*(\pgf@yx/1cm)))}%
}
\makeatother

\begin{document}
\begin{tikzpicture}[
        offtopic/.pic={%
            \calcscaling%
            \contourlength{.5pt} %how thick each copy is
            \node[%
                draw,fill=yellow!50,
                line width=\scalingfactor*2pt,
                rounded corners=\scalingfactor*4pt,
                text=red,font=\scshape\Huge,
                minimum width=\totalwidth cm,
                minimum height=\totalheight cm, 
                inner sep=0pt,
                outer sep=0pt,
            ] (tmp) {\scalebox{1.75}{\contour{black}{Off Topic}}};
            \draw[red,line cap=round,line width=\scalingfactor*3pt] ([shift={(1cm,-.5cm)}]tmp.north west) -- ([shift={(-1cm,-.5cm)}]tmp.north east);    
            \draw[red,line cap=round,line width=\scalingfactor*3pt] ([shift={(1cm,.5cm)}]tmp.south west) -- ([shift={(-1cm,.5cm)}]tmp.south east);
            \draw[red,ultra thick,rounded corners=\scalingfactor*2pt,line width=\scalingfactor*5pt] 
            ([shift={(3pt,-3pt)}]tmp.north west) 
            rectangle 
            ([shift={(-3pt,3pt)}]tmp.south east)
            ;
        }
    ]
    \begin{scope}
    \clip (-5.5,-10) rectangle (5.5,3);
    \foreach \i[%
        evaluate=\i as \myangle using {\i*360/\nn},
        evaluate=\myangle as \myanglestart using {\myangle-3},
    ] in {1,...,\nn} {%
        % \node at (\i,\i) {\i,\myangle,\myanglestart,\deltaangle};
        \def\xx{{\fpeval{cosd{\myangle}}*\radius}}
        \def\yy{{\fpeval{sind{\myangle}}*\radius}}
        % \node at (\i,\i) {\xx,\yy};
        \ifodd\i\relax
            \def\mycolor{green!30}
        \else
            \def\mycolor{magenta!50}
        \fi
        \path[fill=\mycolor] (0,0) -- ++(\xx,\yy) arc[
            start angle=\myanglestart, 
            delta angle={360/\nn}, 
            radius=\radius
        ] -- cycle;
    }
    \end{scope}
    \contourlength{3pt} %how thick each copy is
    \node[font=\Huge\bfseries,align=center] at (0,.5) {\contour{white}{\textsc{Off Topic!}}\\\contour{white}{~~您已脱离\LaTeX{}主题!}};
    \recursivenodes{6}{\totalwidth}{\totalheight}{0}{-3.5}{1}

\end{tikzpicture}
\end{document}

